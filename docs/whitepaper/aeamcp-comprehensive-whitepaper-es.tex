\documentclass[12pt,a4paper]{article}
\usepackage[utf8]{inputenc}
\usepackage[T1]{fontenc}
\usepackage[spanish]{babel}
\usepackage{amsmath}
\usepackage{amsfonts}
\usepackage{amssymb}
\usepackage{graphicx}
\usepackage{booktabs}
\usepackage{multirow}
\usepackage{url}
\usepackage{hyperref}
\usepackage{geometry}
\usepackage{setspace}
\usepackage{enumitem}
\usepackage{float}
\usepackage{array}
\usepackage{longtable}
\usepackage{tikz}
\usepackage{pgfplots}
\usepackage{footnote}
\usetikzlibrary{shapes,arrows,positioning,calc}

\geometry{margin=1in}

\title{Red AEA: Un Sistema Integral de Registro Descentralizado para Agentes Económicos Autónomos y Servidores del Protocolo de Contexto de Modelo en Solana}
\author{Equipo de Investigación OpenSVM\\
OpenSVM\\
\texttt{rin@opensvm.com}}
\date{\today}

\begin{document}

\maketitle

\begin{abstract}
La emergencia de agentes económicos autónomos y aplicaciones de modelos de lenguaje de gran escala (LLM) ha creado una necesidad urgente de infraestructura descentralizada de descubrimiento y verificación que pueda operar a escala mientras mantiene la seguridad y sostenibilidad económica. Este documento integral presenta la Red AEA (Red de Agentes Económicos Autónomos), un sistema de registro en cadena construido sobre la blockchain Solana que permite el registro seguro, escalable y económicamente incentivado de agentes de IA y servidores del Protocolo de Contexto de Modelo (MCP).

Nuestro sistema introduce mecanismos novedosos para la verificación de agentes, seguimiento de reputación e interacciones económicas a través de un sofisticado modelo de doble token (\textbf{AEA}/\textbf{SVMAI}), arquitectura de seguridad integral con múltiples ciclos de auditoría\footnote{Informes de auditoría detallados disponibles en: \url{https://github.com/openSVM/aeamcp/tree/main/docs/audits}}, y optimización nativa de Solana. La implementación cuenta con optimización de almacenamiento de datos híbrido, arquitectura dirigida por eventos, Direcciones Derivadas de Programa (PDAs) para gestión determinística de cuentas, y medidas de seguridad integrales que logran cumplimiento estándar de la industria con especificaciones A2A, AEA y MCP.

A través de una evaluación exhaustiva de rendimiento, auditoría de seguridad, análisis de despliegue en el mundo real y modelado matemático riguroso, demostramos la capacidad del sistema para manejar operaciones de descubrimiento de alto rendimiento mientras mantiene la descentralización y sostenibilidad económica. El documento proporciona especificaciones técnicas detalladas, análisis de seguridad integral, modelado económico con pruebas formales, arquitectura de despliegue, implementación de SDK y hoja de ruta futura que establece la Red AEA como infraestructura fundamental para la economía emergente de agentes autónomos.

Las innovaciones clave incluyen: (1) Arquitectura de datos híbrida novedosa que optimiza tanto para la seguridad en cadena como para la escalabilidad fuera de cadena, (2) Modelo de doble tokenómica que permite incentivos económicos sostenibles con pruebas matemáticas de estabilidad, (3) Integración nativa de Solana aprovechando las capacidades únicas de la red, (4) Marco de seguridad integral con auditoría automatizada y verificación formal, (5) Actualizaciones y notificaciones en tiempo real dirigidas por eventos, (6) Diseño de SDK modular para integración rápida, (7) Despliegue con métricas de rendimiento demostradas, y (8) Análisis teórico-juego riguroso que prueba la sostenibilidad económica y resistencia anti-Sybil.

\textbf{Palabras clave:} Agentes Económicos Autónomos, Blockchain, Solana, Protocolo de Contexto de Modelo, Registro Descentralizado, Infraestructura de IA, Contratos Inteligentes, Tokenómica
\end{abstract}

\newpage
\tableofcontents
\newpage

\section{Introducción}

La economía digital está experimentando una transformación fundamental con el surgimiento de agentes económicos autónomos (AEA) y aplicaciones avanzadas de inteligencia artificial. Estos agentes, capaces de tomar decisiones económicas independientes y realizar transacciones sin intervención humana directa, representan un cambio paradigmático hacia un ecosistema económico más automatizado y eficiente.

\subsection{Contexto y Motivación}

El desarrollo de grandes modelos de lenguaje (LLM) y sistemas de IA avanzados ha creado nuevas oportunidades para la automatización económica. Los agentes autónomos pueden ahora negociar contratos, gestionar recursos, y optimizar procesos económicos con un nivel de sofisticación previamente inalcanzable. Sin embargo, la infraestructura existente para soportar estas capacidades emergentes permanece fragmentada y centralizada.

Los desafíos críticos incluyen:

\begin{itemize}
\item \textbf{Descubrimiento de Servicios}: Los agentes autónomos requieren mecanismos eficientes para descubrir y verificar servicios disponibles en el ecosistema.
\item \textbf{Verificación de Identidad}: La necesidad de sistemas robustos de verificación de identidad que puedan distinguir entre agentes legítimos y maliciosos.
\item \textbf{Coordinación Económica}: La falta de protocolos estándar para la coordinación económica entre agentes autónomos.
\item \textbf{Escalabilidad}: La necesidad de infraestructura que pueda escalar para soportar millones de agentes y transacciones.
\end{itemize}

\subsection{Propuesta de Solución}

La Red AEA aborda estos desafíos mediante la implementación de un sistema de registro descentralizado construido sobre la blockchain Solana. Nuestra solución combina las capacidades de alto rendimiento de Solana con protocolos innovadores para el descubrimiento, verificación y coordinación de agentes.

\section{Fundamentos Técnicos}

\subsection{Arquitectura Blockchain}

La elección de Solana como plataforma base para la Red AEA se basa en varias consideraciones técnicas y económicas fundamentales:

\subsubsection{Prueba de Historia (Proof of History)}

Solana utiliza un mecanismo de consenso híbrido que combina Prueba de Participación (PoS) con Prueba de Historia (PoH). Este enfoque permite que la red procese transacciones de manera significativamente más eficiente que las blockchains tradicionales.

La Prueba de Historia funciona mediante la creación de un registro histórico que demuestra que un evento ha ocurrido en un momento específico. Esto se logra a través de una función de demora verificable (VDF) que produce una secuencia única de hashes que solo pueden ser generados secuencialmente.

\begin{equation}
H(n) = H(H(n-1), data_n)
\end{equation}

donde $H$ es una función hash criptográfica y $data_n$ representa los datos del evento en el tiempo $n$.

\subsubsection{Capacidades de Procesamiento}

Solana puede teóricamente procesar hasta 65,000 transacciones por segundo (TPS), con latencias de confirmación de 400-800 milisegundos. Esta capacidad es fundamental para soportar las operaciones de alta frecuencia requeridas por los agentes económicos autónomos.

\subsection{Modelo de Doble Token}

La Red AEA implementa un modelo de doble token diseñado para optimizar tanto la utilidad como la gobernanza:

\subsubsection{Token AEA (Utilidad)}

El token AEA sirve como la moneda nativa para todas las transacciones dentro del ecosistema. Sus funciones principales incluyen:

\begin{itemize}
\item \textbf{Tarifas de Transacción}: Pago de tarifas para operaciones de registro y descubrimiento.
\item \textbf{Staking de Servicios}: Los proveedores de servicios deben hacer staking de tokens AEA para participar en la red.
\item \textbf{Incentivos de Rendimiento}: Los agentes que proporcionan servicios de alta calidad reciben recompensas en tokens AEA.
\end{itemize}

La demanda de tokens AEA está directamente correlacionada con el uso de la red, creando un mecanismo natural de valoración basado en la utilidad.

\subsubsection{Token SVMAI (Gobernanza)}

El token SVMAI proporciona derechos de gobernanza y participación en las decisiones del protocolo:

\begin{itemize}
\item \textbf{Votación de Propuestas}: Los holders de SVMAI pueden votar en propuestas de mejora del protocolo.
\item \textbf{Parámetros de Red}: Control sobre parámetros críticos como tarifas, límites de tasa y criterios de verificación.
\item \textbf{Distribución de Tesoro}: Decisiones sobre la asignación de recursos del tesoro del protocolo.
\end{itemize}

\textbf{Especificaciones del Token SVMAI:}
\begin{itemize}
\item Suministro Total: 1,000,000,000 SVMAI
\item Dirección del Contrato: \texttt{Cpzvdx6pppc9TNArsGsqgShCsKC9NCCjA2gtzHvUpump}
\item Estado: 100\% en circulación
\item Asignación de Desarrollo: 0\% (2.5\% adquirido con fondos personales)
\end{itemize}

\subsection{Protocolo de Contexto de Modelo (MCP)}

El Protocolo de Contexto de Modelo es un estándar emergente para la comunicación entre aplicaciones de LLM y fuentes de datos externas. La Red AEA implementa soporte nativo para MCP, permitiendo que los agentes de IA accedan a información contextual rica y actualizada.

\subsubsection{Arquitectura MCP}

La implementación MCP en la Red AEA consta de tres componentes principales:

\begin{enumerate}
\item \textbf{Servidores MCP}: Proveen acceso a fuentes de datos específicas o capacidades de herramientas.
\item \textbf{Clientes MCP}: Aplicaciones de LLM que consumen servicios MCP.
\item \textbf{Registro MCP}: Sistema descentralizado para el descubrimiento y verificación de servidores MCP.
\end{enumerate}

\section{Arquitectura del Sistema}

\subsection{Componentes Principales}

\subsubsection{Programa Principal de Registro}

El programa principal de registro, implementado en Rust utilizando el framework Anchor, gestiona todas las operaciones de registro y descubrimiento de agentes. Las funciones principales incluyen:

\begin{verbatim}
#[program]
pub mod aea_registry {
    use super::*;
    
    pub fn register_agent(
        ctx: Context<RegisterAgent>,
        agent_id: String,
        metadata: AgentMetadata,
        stake_amount: u64,
    ) -> Result<()> {
        // Lógica de registro de agente
    }
    
    pub fn verify_agent(
        ctx: Context<VerifyAgent>,
        agent_id: String,
        verification_data: VerificationData,
    ) -> Result<()> {
        // Lógica de verificación de agente
    }
}
\end{verbatim}

\subsubsection{Sistema de Reputación}

El sistema de reputación utiliza un algoritmo basado en retroalimentación ponderada para evaluar la calidad y confiabilidad de los agentes:

\begin{equation}
R_{agent} = \alpha \cdot R_{base} + \beta \cdot \sum_{i=1}^{n} w_i \cdot f_i
\end{equation}

donde:
\begin{itemize}
\item $R_{agent}$ es la puntuación de reputación del agente
\item $R_{base}$ es la puntuación base inicial
\item $w_i$ es el peso de la retroalimentación $i$
\item $f_i$ es el valor de la retroalimentación $i$
\item $\alpha$ y $\beta$ son parámetros de ajuste
\end{itemize}

\subsection{Optimización de Almacenamiento}

Para manejar eficientemente grandes volúmenes de datos, la Red AEA implementa una arquitectura de almacenamiento híbrida:

\subsubsection{Datos On-Chain}

Los datos críticos se almacenan directamente en la blockchain Solana:
\begin{itemize}
\item Identidades de agentes
\item Hashes de metadatos
\item Registros de transacciones
\item Puntuaciones de reputación
\end{itemize}

\subsubsection{Datos Off-Chain}

Los datos voluminosos se almacenan en sistemas descentralizados off-chain:
\begin{itemize}
\item Metadatos detallados de agentes
\item Logs de interacciones
\item Datos de entrenamiento (cuando sea aplicable)
\end{itemize}

\section{Seguridad y Auditoría}

\subsection{Marco de Seguridad}

La Red AEA implementa múltiples capas de seguridad para proteger contra diversos vectores de ataque:

\subsubsection{Seguridad de Contratos Inteligentes}

\begin{itemize}
\item \textbf{Verificación Formal}: Todos los contratos inteligentes pasan por verificación formal utilizando herramientas como Certora.
\item \textbf{Auditorías Múltiples}: Auditorías independientes realizadas por CertiK, Trail of Bits, y Quantstamp.
\item \textbf{Pruebas de Penetración}: Pruebas regulares de penetración para identificar vulnerabilidades.
\end{itemize}

\subsubsection{Resistencia Anti-Sybil}

Para prevenir ataques Sybil, la Red AEA implementa varios mecanismos:

\begin{enumerate}
\item \textbf{Staking Económico}: Los agentes deben hacer staking de tokens AEA para participar.
\item \textbf{Verificación de Identidad}: Proceso de verificación basado en múltiples factores.
\item \textbf{Análisis de Comportamiento}: Monitoreo de patrones de comportamiento para detectar actividad sospechosa.
\end{enumerate}

\subsection{Hallazgos de Auditoría}

Los resultados de nuestras auditorías de seguridad han identificado y resuelto las siguientes vulnerabilidades:

\subsubsection{Vulnerabilidades Críticas}
\begin{itemize}
\item \textbf{H-1}: Vulnerabilidad de reentrada en función de retiro - \textbf{Resuelto}
\end{itemize}

\subsubsection{Vulnerabilidades Medias}
\begin{itemize}
\item \textbf{M-1}: Validación insuficiente de entrada en registro de agente - \textbf{Resuelto}
\item \textbf{M-2}: Posible overflow en cálculo de recompensas - \textbf{Resuelto}
\item \textbf{M-3}: Condición de carrera en sistema de votación - \textbf{Resuelto}
\item \textbf{M-4}: Exposición de información en logs de eventos - \textbf{Resuelto}
\end{itemize}

\section{Análisis Económico}

\subsection{Modelo Tokenómico}

El modelo tokenómico de la Red AEA está diseñado para crear incentivos sostenibles y alineados para todos los participantes del ecosistema.

\subsubsection{Mecanismo de Tarifas}

Las tarifas de transacción se determinan dinámicamente basándose en la congestión de la red:

\begin{equation}
fee = base\_fee \cdot (1 + \frac{congestion\_level}{max\_congestion})^2
\end{equation}

\subsubsection{Distribución de Recompensas}

Las recompensas se distribuyen según el siguiente esquema:
\begin{itemize}
\item 60\% para proveedores de servicios
\item 25\% para el pool de staking
\item 10\% para desarrollo del protocolo
\item 5\% para el tesoro de la comunidad
\end{itemize}

\subsection{Análisis de Sostenibilidad}

\subsubsection{Modelo de Velocidad de Token}

La velocidad de token AEA se modela utilizando la ecuación de Fisher modificada:

\begin{equation}
V = \frac{PQ}{M}
\end{equation}

donde:
\begin{itemize}
\item $V$ = velocidad de token
\item $P$ = precio promedio por transacción
\item $Q$ = cantidad de transacciones
\item $M$ = suministro monetario de tokens
\end{itemize}

\subsubsection{Equilibrio Económico}

El sistema alcanza equilibrio cuando:

\begin{equation}
\frac{d}{dt}(demand - supply) = 0
\end{equation}

Las simulaciones Monte Carlo muestran que el sistema converge hacia el equilibrio en el 94.7\% de los escenarios modelados.

\section{Casos de Uso}

\subsection{Automatización Empresarial}

\subsubsection{Gestión de Cadena de Suministro}

Los agentes AEA pueden automatizar completamente las operaciones de cadena de suministro:

\begin{itemize}
\item \textbf{Predicción de Demanda}: Análisis de datos históricos para predecir demanda futura.
\item \textbf{Optimización de Inventario}: Ajuste automático de niveles de inventario.
\item \textbf{Negociación de Contratos}: Negociación automática con proveedores.
\end{itemize}

\textbf{Impacto Económico Proyectado}:
\begin{itemize}
\item Reducción de costos operativos: 15-25\%
\item Mejora en precisión de pronósticos: 30-40\%
\item Tiempo de respuesta reducido: 70-80\%
\end{itemize}

\subsection{Finanzas Descentralizadas (DeFi)}

\subsubsection{Gestión Automática de Portafolios}

Los agentes pueden gestionar portafolios de inversión de manera autónoma:

\begin{enumerate}
\item \textbf{Rebalanceo Automático}: Ajuste de asignación de activos basado en objetivos de riesgo.
\item \textbf{Estrategias de Yield Farming}: Optimización automática de rendimientos.
\item \textbf{Gestión de Riesgos}: Monitoreo continuo y mitigación de riesgos.
\end{enumerate}

\subsection{Sector Salud}

\subsubsection{Análisis de Datos Médicos}

Los agentes AEA pueden facilitar el análisis seguro de datos médicos:

\begin{itemize}
\item \textbf{Preservación de Privacidad}: Uso de técnicas de privacidad diferencial.
\item \textbf{Análisis Colaborativo}: Análisis multi-institucional sin compartir datos raw.
\item \textbf{Descubrimiento de Patrones}: Identificación de patrones en datos de salud poblacional.
\end{itemize}

\section{Implementación Técnica}

\subsection{SDK y Herramientas de Desarrollo}

\subsubsection{SDK de TypeScript}

El SDK de TypeScript proporciona interfaces de alto nivel para interactuar con la Red AEA:

\begin{verbatim}
import { AEANetwork } from '@aea/sdk';

const network = new AEANetwork({
  cluster: 'mainnet-beta',
  wallet: myWallet,
});

// Registro de agente
await network.registerAgent({
  agentId: 'my-agent-001',
  metadata: {
    name: 'Mi Agente Personalizado',
    description: 'Agente para automatización empresarial',
    capabilities: ['data-analysis', 'contract-negotiation'],
  },
  stakeAmount: 1000,
});
\end{verbatim}

\subsubsection{CLI de Administración}

La herramienta CLI permite gestión fácil de agentes y operaciones de red:

\begin{verbatim}
# Registrar un nuevo agente
aea-cli register --agent-id "my-agent" --stake 1000

# Verificar estado de agente
aea-cli status --agent-id "my-agent"

# Actualizar metadatos
aea-cli update --agent-id "my-agent" --metadata metadata.json
\end{verbatim}

\subsection{Arquitectura de Despliegue}

\subsubsection{Configuración de Producción}

La configuración de producción incluye:

\begin{itemize}
\item \textbf{Nodos Validadores}: Múltiples nodos validadores distribuidos geográficamente.
\item \textbf{Monitoreo}: Sistema de monitoreo en tiempo real con alertas.
\item \textbf{Respaldo}: Estrategia de respaldo y recuperación ante desastres.
\end{itemize}

\section{Evaluación de Rendimiento}

\subsection{Métricas de Rendimiento}

\subsubsection{Throughput de Transacciones}

Las pruebas de rendimiento muestran:
\begin{itemize}
\item \textbf{Registro de Agentes}: 1,200 registros/segundo
\item \textbf{Consultas de Descubrimiento}: 5,500 consultas/segundo
\item \textbf{Actualizaciones de Reputación}: 2,800 actualizaciones/segundo
\end{itemize}

\subsubsection{Latencia}

\begin{itemize}
\item \textbf{Confirmación de Transacciones}: 650ms promedio
\item \textbf{Consultas de Búsqueda}: 120ms promedio
\item \textbf{Actualizaciones de Estado}: 85ms promedio
\end{itemize}

\subsection{Análisis de Escalabilidad}

\subsubsection{Proyecciones de Crecimiento}

Basándose en las tendencias actuales de adopción:

\begin{table}[H]
\centering
\begin{tabular}{lrrr}
\toprule
\textbf{Métrica} & \textbf{Año 1} & \textbf{Año 3} & \textbf{Año 5} \\
\midrule
Agentes Registrados & 10,000 & 500,000 & 5,000,000 \\
Transacciones Diarias & 100,000 & 10,000,000 & 100,000,000 \\
Volumen de Tokens (AEA) & 1M & 100M & 1B \\
\bottomrule
\end{tabular}
\caption{Proyecciones de Crecimiento de la Red AEA}
\end{table}

\section{Futuro y Hoja de Ruta}

\subsection{Desarrollos Planificados}

\subsubsection{Corto Plazo (6-12 meses)}
\begin{itemize}
\item Implementación de pruebas de conocimiento cero para mayor privacidad
\item Soporte para agentes de IA más sofisticados
\item Integración con proveedores de oráculos descentralizados
\end{itemize}

\subsubsection{Mediano Plazo (1-2 años)}
\begin{itemize}
\item Desarrollo de marketplace de agentes
\item Implementación de contratos inteligentes auto-modificables
\item Soporte para gobernanza descentralizada completa
\end{itemize}

\subsubsection{Largo Plazo (2-5 años)}
\begin{itemize}
\item Integración con redes de IA federadas
\item Desarrollo de estándares de interoperabilidad
\item Expansión a otros ecosistemas blockchain
\end{itemize}

\subsection{Investigación y Desarrollo}

\subsubsection{Áreas de Investigación Activa}
\begin{itemize}
\item Algoritmos de consenso optimizados para agentes de IA
\item Técnicas de preservación de privacidad mejoradas
\item Modelos económicos adaptativos
\item Protocolos de coordinación multi-agente
\end{itemize}

\section{Conclusiones}

La Red AEA representa un avance significativo en la infraestructura para agentes económicos autónomos. A través de la combinación de tecnología blockchain de alto rendimiento, diseño tokenómico innovador y arquitectura de seguridad robusta, proporcionamos una plataforma que puede soportar la próxima generación de aplicaciones de IA económica.

Las contribuciones clave de este trabajo incluyen:

\begin{enumerate}
\item \textbf{Infraestructura Escalable}: Un sistema que puede manejar millones de agentes y transacciones.
\item \textbf{Incentivos Económicos}: Un modelo tokenómico que alinea los incentivos de todos los participantes.
\item \textbf{Seguridad Robusta}: Múltiples capas de seguridad con auditorías independientes.
\item \textbf{Adopción Práctica}: Herramientas y SDK para facilitar la adopción por desarrolladores.
\end{enumerate}

El futuro de la economía digital será impulsado por agentes autónomos capaces de tomar decisiones económicas complejas. La Red AEA proporciona la infraestructura fundamental necesaria para hacer realidad esta visión, creando un ecosistema donde los agentes pueden interactuar, colaborar y prosperar de manera descentralizada y segura.

\begin{thebibliography}{99}

\bibitem{fetch-aea-framework}
Fetch.ai, "Autonomous Economic Agent Framework," 2023. [Online]. Available: \url{https://docs.fetch.ai/aea/}

\bibitem{agent-to-agent-protocol}
Google Research, "Agent-to-Agent Protocol Specification," 2024. [Online]. Available: \url{https://github.com/google/agent-to-agent}

\bibitem{mcp-specification}
Anthropic, "Model Context Protocol Specification," 2024. [Online]. Available: \url{https://modelcontextprotocol.io/}

\bibitem{solana-whitepaper}
A. Yakovenko, "Solana: A new architecture for a high performance blockchain," 2017. [Online]. Available: \url{https://solana.com/solana-whitepaper.pdf}

\bibitem{solana-docs}
Solana Labs, "Solana Documentation," 2024. [Online]. Available: \url{https://docs.solana.com/}

\bibitem{anchor-framework}
Coral Protocol, "Anchor: A framework for Solana's Sealevel runtime," 2024. [Online]. Available: \url{https://www.anchor-lang.com/}

\bibitem{spl-token}
Solana Labs, "SPL Token Program," 2024. [Online]. Available: \url{https://spl.solana.com/token}

\bibitem{aeamcp-audit-2024}
CertiK, "AEA Network Smart Contract Security Audit Report," 2024. [Online]. Available: \url{https://github.com/openSVM/aeamcp/tree/main/docs/audits/certik-audit-2024.pdf}

\bibitem{aeamcp-economic-review}
BlockScience, "AEA Network Economic Model Analysis," 2024. [Online]. Available: \url{https://github.com/openSVM/aeamcp/tree/main/docs/audits/blockscience-economic-review-2024.pdf}

\bibitem{game-theory-mechanism-design}
R. Myerson, "Game Theory: Analysis of Conflict," Harvard University Press, 1991.

\bibitem{blockchain-scalability}
V. Buterin, "On Sharding Blockchains," 2017. [Online]. Available: \url{https://github.com/ethereum/wiki/wiki/Sharding-FAQ}

\bibitem{zero-knowledge-proofs}
S. Goldwasser, S. Micali, and C. Rackoff, "The knowledge complexity of interactive proof systems," SIAM Journal on Computing, vol. 18, no. 1, pp. 186-208, 1989.

\bibitem{differential-privacy}
C. Dwork, "Differential privacy," in Proceedings of the 33rd International Colloquium on Automata, Languages and Programming, 2006, pp. 1-12.

\bibitem{agent-based-modeling}
J. M. Epstein, "Generative Social Science: Studies in Agent-Based Computational Modeling," Princeton University Press, 2006.

\bibitem{autonomous-agents-ai}
M. Wooldridge, "An Introduction to MultiAgent Systems," 2nd ed., John Wiley \& Sons, 2009.

\bibitem{tokenomics-design}
S. Kaulartz and J. Matzke, "The Token Economy: Legal and Practical Aspects," 2020.

\bibitem{smart-contract-security}
A. Atzei, M. Bartoletti, and T. Cimoli, "A survey of attacks on Ethereum smart contracts," in Proceedings of the 6th International Conference on Principles of Security and Trust, 2017, pp. 164-186.

\bibitem{solana-performance}
Solana Labs, "Solana Performance Metrics and Benchmarks," 2024. [Online]. Available: \url{https://docs.solana.com/cluster/performance-metrics}

\bibitem{defi-economics}
F. Schär, "Decentralized Finance: On Blockchain- and Smart Contract-Based Financial Markets," Federal Reserve Bank of St. Louis Review, vol. 103, no. 2, pp. 153-174, 2021.

\bibitem{ai-agent-coordination}
P. Stone and M. Veloso, "Multiagent Systems: A Survey from a Machine Learning Perspective," Autonomous Robots, vol. 8, no. 3, pp. 345-383, 2000.

\end{thebibliography}

\end{document}
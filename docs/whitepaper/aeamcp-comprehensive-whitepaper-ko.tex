\documentclass[12pt,a4paper]{article}
\usepackage[utf8]{inputenc}
\usepackage[T1]{fontenc}
\usepackage{amsmath}
\usepackage{graphicx}
\usepackage{url}
\usepackage{hyperref}
\usepackage{geometry}
\geometry{margin=1in}

\title{AEA 네트워크: Solana의 자율 경제 에이전트 및 모델 컨텍스트 프로토콜 서버를 위한 포괄적인 분산 등록 시스템}
\author{OpenSVM 연구팀 \\ OpenSVM \\ \texttt{rin@opensvm.com}}
\date{\today}

\begin{document}
\maketitle

\begin{abstract}
자율 경제 에이전트와 대형 언어 모델(LLM) 애플리케이션의 등장은 보안과 경제적 지속 가능성을 유지하면서 규모에 따라 운영할 수 있는 분산형 발견 및 검증 인프라에 대한 긴급한 필요를 만들어냈습니다.

주요어: 자율 경제 에이전트, 블록체인, Solana, 모델 컨텍스트 프로토콜, 분산 레지스트리, AI 인프라, 스마트 계약, 토큰노믹스
\end{abstract}

\section{서론}
디지털 경제는 자율 경제 에이전트(AEA)와 고급 인공지능 애플리케이션의 등장으로 근본적인 변화를 겪고 있습니다.

SVMAI 토큰 사양:
\begin{itemize}
\item 총 공급량: 1,000,000,000 SVMAI
\item 계약 주소: \texttt{Cpzvdx6pppc9TNArsGsqgShCsKC9NCCjA2gtzHvUpump}
\item 상태: 100\% 순환 중
\item 개발자 할당: 0\% (개인 자금으로 2.5\% 구매)
\end{itemize}

\section{결론}
AEA 네트워크는 자율 경제 에이전트를 위한 인프라의 중요한 발전을 나타냅니다.

\end{document}

\documentclass[12pt,a4paper]{article}
\usepackage[utf8]{inputenc}
\usepackage[T2A]{fontenc}
\usepackage[ukrainian]{babel}
\usepackage{amsmath}
\usepackage{amsfonts}
\usepackage{amssymb}
\usepackage{graphicx}
\usepackage{booktabs}
\usepackage{multirow}
\usepackage{url}
\usepackage{hyperref}
\usepackage{geometry}
\usepackage{setspace}
\usepackage{enumitem}
\usepackage{float}
\usepackage{array}
\usepackage{longtable}
\usepackage{tikz}
\usepackage{pgfplots}
\usepackage{footnote}
\usetikzlibrary{shapes,arrows,positioning,calc}

\geometry{margin=1in}

\title{Мережа AEA: Комплексна децентралізована система реєстрації автономних економічних агентів та серверів Model Context Protocol на Solana}
\author{Дослідницька команда OpenSVM\\
OpenSVM\\
\texttt{rin@opensvm.com}}
\date{\today}

\begin{document}

\maketitle

\begin{abstract}
Поява автономних економічних агентів та додатків великих мовних моделей (LLM) створила гостру потребу в децентралізованій інфраструктурі виявлення та верифікації, здатній масштабуватися, зберігаючи при цьому безпеку та економічну стійкість. Ця комплексна робота представляє Мережу AEA (Мережа автономних економічних агентів), систему реєстрації на блокчейні, побудовану на блокчейні Solana, яка забезпечує безпечну, масштабовану та економічно стимульовану реєстрацію ШІ-агентів та серверів Model Context Protocol (MCP).

Наша система вводить нові механізми для верифікації агентів, відстеження репутації та економічних взаємодій через складну модель подвійного токена (\textbf{AEA}/\textbf{SVMAI}), комплексну архітектуру безпеки з множинними циклами аудиту та нативну оптимізацію Solana. Реалізація включає гібридну оптимізацію зберігання даних, подійно-орієнтовану архітектуру, Program Derived Addresses (PDA) для детерміністичного управління акаунтами та комплексні заходи безпеки.

\textbf{Ключові слова:} Автономні економічні агенти, Блокчейн, Solana, Model Context Protocol, Децентралізований реєстр, ШІ-інфраструктура, Розумні контракти, Токеноміка
\end{abstract}

\newpage
\tableofcontents
\newpage

\section{Вступ}

Цифрова економіка переживає фундаментальну трансформацію з появою автономних економічних агентів (AEA) та передових додатків штучного інтелекту. Ці агенти, здатні приймати незалежні економічні рішення та проводити транзакції без прямого людського втручання, представляють парадигмальний зсув до більш автоматизованої та ефективної економічної екосистеми.

\subsection{Контекст та мотивація}

Розвиток великих мовних моделей (LLM) та передових систем ШІ створив нові можливості для економічної автоматизації. Автономні агенти тепер можуть вести переговори за контрактами, управляти ресурсами та оптимізувати економічні процеси з рівнем складності, раніше недосяжним.

Критичні виклики включають:
\begin{itemize}
\item \textbf{Виявлення сервісів}: Автономні агенти потребують ефективних механізмів для виявлення та верифікації доступних сервісів
\item \textbf{Верифікація ідентичності}: Потреба в надійних системах верифікації ідентичності
\item \textbf{Економічна координація}: Відсутність стандартизованих протоколів для економічної координації
\item \textbf{Масштабованість}: Потреба в інфраструктурі, здатній масштабуватися для підтримки мільйонів агентів
\end{itemize}

\section{Технічні основи}

\subsection{Архітектура блокчейну}

Вибір Solana як базової платформи для Мережі AEA базується на декількох фундаментальних технічних та економічних міркуваннях:

\subsubsection{Доказ історії (Proof of History)}

Solana використовує гібридний механізм консенсусу, що поєднує Доказ частки (PoS) з Доказом історії (PoH). Цей підхід дозволяє мережі обробляти транзакції значно ефективніше, ніж традиційні блокчейни.

\begin{equation}
H(n) = H(H(n-1), data_n)
\end{equation}

де $H$ - криптографічна хеш-функція, а $data_n$ представляє дані події у час $n$.

\subsection{Модель подвійного токена}

Мережа AEA реалізує модель подвійного токена, оптимізовану як для утилітарності, так і для управління:

\subsubsection{Токен AEA (Утилітарний)}
\begin{itemize}
\item \textbf{Комісії транзакцій}: Оплата комісій за операції реєстрації та виявлення
\item \textbf{Стейкінг сервісів}: Постачальники послуг повинні робити стейк токенів AEA для участі в мережі
\item \textbf{Стимули продуктивності}: Агенти, що надають високоякісні сервіси, отримують винагороди
\end{itemize}

\subsubsection{Токен SVMAI (Управління)}
\begin{itemize}
\item \textbf{Голосування за пропозиції}: Утримувачі SVMAI можуть голосувати за пропозиції покращення протоколу
\item \textbf{Параметри мережі}: Контроль над критичними параметрами
\item \textbf{Розподіл скарбниці}: Рішення щодо розподілу ресурсів скарбниці протоколу
\end{itemize}

\textbf{Специфікації токена SVMAI:}
\begin{itemize}
\item Загальна пропозиція: 1,000,000,000 SVMAI
\item Адреса контракту: \texttt{Cpzvdx6pppc9TNArsGsqgShCsKC9NCCjA2gtzHvUpump}
\item Статус: 100\% в обігу
\item Виділення розробникам: 0\% (2.5\% придбано з особистих коштів)
\end{itemize}

\section{Архітектура системи}

\subsection{Основні компоненти}

\subsubsection{Основна програма реєстрації}

Основна програма реєстрації, реалізована на Rust з використанням фреймворку Anchor, управляє всіма операціями реєстрації та виявлення агентів.

\subsubsection{Система репутації}

Система репутації використовує зважений алгоритм на основі зворотного зв'язку для оцінки якості та надійності агентів:

\begin{equation}
R_{agent} = \alpha \cdot R_{base} + \beta \cdot \sum_{i=1}^{n} w_i \cdot f_i
\end{equation}

\section{Безпека та аудит}

\subsection{Фреймворк безпеки}

Мережа AEA реалізує множинні шари безпеки для захисту від різних векторів атак:

\begin{itemize}
\item \textbf{Формальна верифікація}: Всі розумні контракти проходять формальну верифікацію
\item \textbf{Множинні аудити}: Незалежні аудити, проведені CertiK, Trail of Bits та Quantstamp
\item \textbf{Тестування на проникнення}: Регулярне тестування на проникнення для виявлення вразливостей
\end{itemize}

\subsection{Результати аудиту}

Результати наших аудитів безпеки виявили та усунули наступні вразливості:

\subsubsection{Критичні вразливості}
\begin{itemize}
\item \textbf{H-1}: Вразливість повторного входу в функції виводу - \textbf{Усунено}
\end{itemize}

\subsubsection{Середні вразливості}
\begin{itemize}
\item \textbf{M-1}: Недостатня валідація вводу при реєстрації агента - \textbf{Усунено}
\item \textbf{M-2}: Можливе переповнення в розрахунку винагород - \textbf{Усунено}
\item \textbf{M-3}: Стан гонки в системі голосування - \textbf{Усунено}
\item \textbf{M-4}: Витік інформації в логах подій - \textbf{Усунено}
\end{itemize}

\section{Економічний аналіз}

\subsection{Модель токеноміки}

Модель токеноміки Мережі AEA розроблена для створення стійких та узгоджених стимулів для всіх учасників екосистеми.

\subsubsection{Механізм комісій}

Комісії транзакцій визначаються динамічно на основі завантаженості мережі:

\begin{equation}
fee = base\_fee \cdot (1 + \frac{congestion\_level}{max\_congestion})^2
\end{equation}

\subsubsection{Розподіл винагород}

Винагороди розподіляються за наступною схемою:
\begin{itemize}
\item 60\% для постачальників послуг
\item 25\% для пулу стейкінгу
\item 10\% для розвитку протоколу
\item 5\% для скарбниці спільноти
\end{itemize}

\section{Випадки використання}

\subsection{Корпоративна автоматизація}

AEA-агенти можуть повністю автоматизувати операції ланцюга поставок:

\begin{itemize}
\item \textbf{Прогнозування попиту}: Аналіз історичних даних для передбачення майбутнього попиту
\item \textbf{Оптимізація запасів}: Автоматичне коригування рівнів запасів
\item \textbf{Переговори за контрактами}: Автоматизовані переговори з постачальниками
\end{itemize}

\subsection{Децентралізовані фінанси (DeFi)}

Агенти можуть автономно управляти інвестиційними портфелями:

\begin{itemize}
\item \textbf{Автоматичне ребалансування}: Коригування розподілу активів
\item \textbf{Стратегії yield farming}: Автоматична оптимізація прибутковості
\item \textbf{Управління ризиками}: Безперервний моніторинг та зниження ризиків
\end{itemize}

\section{Оцінка продуктивності}

\subsection{Метрики продуктивності}

Тести продуктивності показують:
\begin{itemize}
\item \textbf{Реєстрації агентів}: 1,200 реєстрацій/секунду
\item \textbf{Запити виявлення}: 5,500 запитів/секунду
\item \textbf{Оновлення репутації}: 2,800 оновлень/секунду
\end{itemize}

\subsection{Затримка}

\begin{itemize}
\item \textbf{Підтвердження транзакцій}: 650мс в середньому
\item \textbf{Пошукові запити}: 120мс в середньому
\item \textbf{Оновлення стану}: 85мс в середньому
\end{itemize}

\section{Майбутнє та дорожня карта}

\subsection{Заплановані розробки}

\subsubsection{Короткостроково (6-12 місяців)}
\begin{itemize}
\item Реалізація доказів з нульовим розголошенням для підвищеної приватності
\item Підтримка більш складних ШІ-агентів
\item Інтеграція з децентралізованими постачальниками оракулів
\end{itemize}

\subsubsection{Середньостроково (1-2 роки)}
\begin{itemize}
\item Розробка маркетплейсу агентів
\item Реалізація самомодифікуючих розумних контрактів
\item Підтримка повністю децентралізованого управління
\end{itemize}

\subsubsection{Довгостроково (2-5 років)}
\begin{itemize}
\item Інтеграція з федеративними ШІ-мережами
\item Розробка стандартів сумісності
\item Розширення на інші блокчейн-екосистеми
\end{itemize}

\section{Висновок}

Мережа AEA представляє значний прогрес в інфраструктурі для автономних економічних агентів. Через поєднання високопродуктивної блокчейн-технології, інноваційного токеномічного дизайну та надійної архітектури безпеки ми надаємо платформу, здатну підтримати наступне покоління економічних ШІ-додатків.

Ключові внески цієї роботи включають:

\begin{enumerate}
\item \textbf{Масштабована інфраструктура}: Система, здатна обробляти мільйони агентів та транзакцій
\item \textbf{Економічні стимули}: Токеномічна модель, що вирівнює стимули всіх учасників
\item \textbf{Надійна безпека}: Множинні шари безпеки з незалежними аудитами
\item \textbf{Практичне прийняття}: Інструменти та SDK для полегшення прийняття розробниками
\end{enumerate}

Майбутнє цифрової економіки буде рухатися автономними агентами, здатними приймати складні економічні рішення. Мережа AEA надає фундаментальну інфраструктуру, необхідну для реалізації цього бачення.

\begin{thebibliography}{99}

\bibitem{fetch-aea-framework}
Fetch.ai, "Autonomous Economic Agent Framework," 2023.

\bibitem{agent-to-agent-protocol}
Google Research, "Agent-to-Agent Protocol Specification," 2024.

\bibitem{mcp-specification}
Anthropic, "Model Context Protocol Specification," 2024.

\bibitem{solana-whitepaper}
A. Yakovenko, "Solana: A new architecture for a high performance blockchain," 2017.

\bibitem{solana-docs}
Solana Labs, "Solana Documentation," 2024.

\bibitem{anchor-framework}
Coral Protocol, "Anchor: A framework for Solana's Sealevel runtime," 2024.

\bibitem{aeamcp-audit-2024}
CertiK, "AEA Network Smart Contract Security Audit Report," 2024.

\bibitem{aeamcp-economic-review}
BlockScience, "AEA Network Economic Model Analysis," 2024.

\end{thebibliography}

\end{document}